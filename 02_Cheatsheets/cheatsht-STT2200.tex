\documentclass[10pt, french]{article}
%% -----------------------------
%% Préambule
%% -----------------------------
\input{cheatsht-preamble-general.tex}
%% -----------------------------
%% Redefine from template
%% -----------------------------
\def\auteur{Gabriel Crépeault-Cauchon}
%% -----------------------------
%% Variable definition
%% -----------------------------
\def\cours{Analyse de données}
\def\sigle{STT-2200}
%% -----------------------------
%% Colour setup for sections
%% -----------------------------
\def\SectionColor{cobalt}
\def\SubSectionColor{cobalt!50!black}
\def\SubSubSection{cobalt!75!black}

% 
% Débute numérotation où on est rendu pour l'examen final
% 
\setcounter{section}{9}

%% -----------------------------
%% Début du document
%% -----------------------------
\begin{document}

\begin{multicols*}{2}
\section{Analyse discriminante}

\subsection{Méthode de Fisher}

\subsection{Analyse discriminante linéaire}


\subsection{Analyse discriminante quadratique}


\subsection{Matrice de confusion}
Pour tout modèle de classification, on peut construire un tableau de mauvaisse classification (\textit{confusion matrix} en anglais), tel que

\begin{tabular}{|c|c|c|}
\hline
Réel/Prédit & Positif & Négatif \\
\hline
Positif & VP & FN \\
\hline
Négatif & FP & FN \\
\hline
\end{tabular}

Et on peut aussi calculer quelques mesures de performance du modèle : 

\begin{align*}
\text{Accuracy}	& = \frac{VP + VN}{\text{\# total obs.}} \\
\text{Sensitivity}	& = \frac{VP}{\text{\# total positive}} = \frac{VP}{VP + FN} \\
\text{Specificity}	& = \frac{VN}{\text{\# total negative}} = \frac{VN}{VN + FP} \\
\text{Precision} & = \frac{VP}{VP + FP} \\
\text{Score F1}	& = \frac{2}{\frac{1}{\text{Sensitivity}} + \frac{1}{\text{Specificity}}} = \frac{2 VP}{2 VP + FP + FN} \\
\end{align*}




\section{Arbres de classification}

\subsection{Approche}



\subsection{Méthodes d'ensemble}






\end{multicols*}
%% -----------------------------
%% Fin du document
%% -----------------------------
\end{document}
