\documentclass[10pt, french]{article}
\usepackage[landscape, hmargin=1cm, vmargin=1.7cm]{geometry}

%% -----------------------------
%% Préambule
%% -----------------------------
\input{cheatsht-preamble-general.tex}
%% -----------------------------
%% Variable definition
%% -----------------------------
%% -----------------------------
%% Footer and header customization
%% -----------------------------
\def\cours{Mathématiques financières}
\def\sigle{ACT-1001}
\fancyfoot[R]{\thepage ~de~ \pageref{LastPage}}
\setlist{leftmargin=*}


%% -----------------------------
%% Colour setup for sections
%% -----------------------------
\def\SectionColor{green!50!black}
\def\SubSectionColor{green!20!black}
%% -----------------------------
%% Definition of LaTex math commands
%% -----------------------------
\newcommand{\norm}[1]{\left\lVert#1\right\rVert}
%
% Redefine authors
%
\definecolor{burgundy}{rgb}{0.5, 0.0, 0.13}

\begin{document}
\begin{center}
	\textsc{\Large Contributeurs}\\[0.5cm] 
\end{center}
\begin{contrib}{ACT-1002\: Analyse probabiliste des risques actuariels}
    \begin{description}
        \item[aut.] Alec James van Rassel
        \item[aut., cre.] Nicolas Chevrette
        \item[src.] Thomas Landry
    \end{description}
    \end{contrib}

\newpage

\raggedcolumns
\begin{multicols*}{3}

\section{Chapitre 1: Mesure du taux d'intérêt}
\begin{mathfinch1}{La mesure du taux d'intérêt}
\begin{description}
    \item[Intérêt simple :] L'intérêt simple correspond à un montant constant qui s'ajoute à votre investissement sur une certaine \\
    période de temps. Par exemple, 100\$ à un taux d'intérêt simple de 5\% rapporterait 5\$ à chaque année. \\
    Pour savoir la quantité d'argent accumulé après n années, la formule utilisée est la suivante (où C correspond à l'investissement initial): \\
    $C(1 + ni)$
\end{description}
\begin{description}
    \item[Intérêt composé :] L'intérêt composé est un montant qui varie dans le temps, car celui-ci se réapplique constamment sur la somme \\
    que j'ai à un moment donnée. Par exemple, 100\$ à un taux d'intérêt de 5\% rapporterait 5\$ la première année. \\
    Après un an, j'aurais donc 105\$ en poche. Pour la deuxième année, 5\% d'intérêt s'appliquerait sur mon 105\$ et non sur 100\$. \\
    Ainsi, je gagnerais $5\% * 105\$ = 5,25\$$ la deuxième année pour un total accumulé de 110,25\$ après deux ans. \\
    La formule pour calculer la quantité d'argent accumulé après n années à un taux d'intérêt composé i est la suivante : $ 𝐶(1 + i)^{n}$
\end{description}

\end{mathfinch1}

\end{multicols*}

\end{document}
