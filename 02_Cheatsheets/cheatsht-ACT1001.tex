\documentclass[10pt, french]{article}
\usepackage[landscape, hmargin=1cm, vmargin=1.7cm]{geometry}

%% -----------------------------
%% Préambule
%% -----------------------------
\input{cheatsht-preamble-general.tex}
%% -----------------------------
%% Variable definition
%% -----------------------------
%% -----------------------------
%% Footer and header customization
%% -----------------------------
\def\cours{Mathématiques financières}
\def\sigle{ACT-1001}
\fancyfoot[R]{\thepage ~de~ \pageref{LastPage}}
\setlist{leftmargin=*}


%% -----------------------------
%% Colour setup for sections
%% -----------------------------
\def\SectionColor{green!50!black}
\def\SubSectionColor{green!20!black}
%% -----------------------------
%% Definition of LaTex math commands
%% -----------------------------
\newcommand{\norm}[1]{\left\lVert#1\right\rVert}
%
% Redefine authors
%
\definecolor{burgundy}{rgb}{0.5, 0.0, 0.13}

\begin{document}
\begin{center}
	\textsc{\Large Contributeurs}\\[0.5cm] 
\end{center}
\begin{contrib}{ACT-1002\: Analyse probabiliste des risques actuariels}
    \begin{description}
        \item[aut.] Alec James van Rassel
        \item[aut., cre.] Nicolas Chevrette
        \item[src.] Thomas Landry
    \end{description}
    \end{contrib}

\newpage

\raggedcolumns
\begin{multicols*}{3}

\section{Chapitre 1: Mesure du taux d'intérêt}
\begin{mathfinch1}{La mesure du taux d'intérêt}
\begin{description}
    \item[Intérêt simple :] L'intérêt simple correspond à un montant constant qui s'ajoute à votre investissement sur une certaine 
    période de temps. Par exemple, 100\$ à un taux d'intérêt simple annuel de 5\% rapporterait 5\$ à chaque année. 
    Pour savoir la quantité d'argent accumulé après un certain nombre de temps $t$, la formule utilisée est la suivante (où A(0) correspond à l'investissement initial): 
    $A(t) = A(0)a(t) = A(0)(1 + ti)$, car $a(t) = (1 + ti)$ pour un taux d'intérêt simple. 
\end{description}
\begin{description}
    \item[Intérêt composé :] L'intérêt composé est un montant qui varie dans le temps, car celui-ci se réapplique constamment sur la somme 
    que j'ai à un moment donnée. Par exemple, 100\$ à un taux d'intérêt de 5\% rapporterait 5\$ la première année. 
    Après un an, j'aurais donc 105\$ en poche. Pour la deuxième année, 5\% d'intérêt s'appliquerait sur mon 105\$ et non sur 100\$.
    Ainsi, je gagnerais $5\% * 105\$ = 5,25\$$ la deuxième année pour un total accumulé de 110,25\$ après deux ans. 
    La formule pour calculer la quantité d'argent accumulé après un certain nombre de temps à un taux d'intérêt composé annuel i est la suivante : 
    $A(t) = A(0)a(t) = A(0)(1 + i)^{n}$, car $a(t) =(1 + i)^{n}$ pour un taux d'intérêt composé.
\end{description}
\begin{description} % changer le format de l'exemple
    \item[Taux de rendement annuel moyen :] Pour trouver, un taux de rendement effectif sur n années, simplement multiplier les taux de rendement entre eux. 
    Pour obtenir un taux de rendement moyen sur n années, il faut faire la racine à la n du taux de rendement trouvé. Exemple : on pose 6\% d'intérêt la première année,
    7\% d'intérêt la deuxième année, 8\% d'intérêt la troisième année. Le taux d'intérêt effectif sur trois ans serait $(1.06)(1.07)(1.08) - 1 =  0.22$. 
    Le taux d'intérêt moyen serait $∖sqrt[3]{1 + 0.22} - 1 = 0.069$.
\end{description}
\begin{description}%remplir la réponse chui trop lache en live 
    \item[Trouver le taux d'intérêt à partir d'une fonction d'accumulation :] Un cas particulier qui pourrait survenir est de trouver le taux d'intérêt sur une période 
    de temps à l'aide d'une fonction d'accumulation. Par exemple, $a(t) = 0.001t^2 + 0.05t + 1$ et on veut savoir le taux d'intérêt entre $t=2$ et $t=3$. Il faut appliquer la formule
    $i_{t+h} = \frac{a(t+h)}{a(t)} - 1 $, ce qui dans l'exemple donnera $\frac{0.001(t+h)^2 + 0.05(t+h) + 1}{0.001t^2 + 0.05t + 1} - 1 = \frac{0.001*3^2 + 0.05*3 + 1}{0.001*2^2 + 0.05*2 + 1} - 1 = À REMPLIR$
\end{description}
\begin{description} % rajouter graphique? + rajouter exemple
    \item[Équation de valeur :] En mathématiques financières,  la plupart des questions se résument à trouver $x$ dans une équation. 
    Premièrement, faire une ligne du temps peut être très utile. On choisit un moment sur la ligne de temps (si le moment n'est pas spécifié bien sûr).
    Par la suite, on actualise (diviser par $(1 + i)^t$) tous les montants après cette date et on projète dans le futur tous les montants avant (multiplier par $(1 + i)^t)$. 
    Il peut être préférable de choisir le moment $t = 0$ et d'actualiser tous les montants dans certains cas. Par la suite, on met les entrées d'argent d'un côté et les sorties de l'autre. 
    On résout algébriquement pour trouver $x$.
\end{description}
\begin{description} 
    \item[Taux d'intérêt nominal :] Le taux d'intérêt nominal s'applique lorsque la période d'intérêt est plus petite qu'un an et correspond au taux d'intérêt multiplié par le nombre de périodes
    Par exemple, si vous payez de l'intérêt à chaque 6 mois à 3\%, le taux d'intérêt nominal sera de $3\% * 2 = 6\%$. On voit que ce n'est pas égal au taux d'intérêt composé, 
    qui correspond à $1.03^2 - 1 = 6.09\%$. Dès que vous voyez "nominal", changer ça en taux d'intérêt effectif à l'aide de l'équation :$i = (1 + \frac{i^(m)}{m})^m - 1$. 
    Cette équation permet de remettre le taux d'intérêt sur une période de 6 mois par exemple, pour ensuite le remettre sur une période d'un an à l'aide d'une puissance et 
    non d'une mutliplication (car l'intérêt composé s'applique de cette façon). Il est important de noter que le taux d'intéret effectif sera toujours supérieur au taux 
    d'intérêt nominal.
\end{description}
\begin{description} 
    \item[Taux d'escompte :] Le taux d'escompte correspond au pourcentage qu'il faut retrancher à la valeur accumulé après un certain temps afin de revenir à
    l'investissement initial. Par exemple, pour un investissement d'un an, $A(0) = A(1) - dA(1) = (1 - d)A(1)$. Il peut être intéressant de se rappeler de la formule
    du taux d'escompte simple $A(0) = (1 - d)A(n)$. Par contre, pour le taux d'escompte composé, il est préférable de travailler avec le taux d'intérêt avec la formule 
    $i = \frac{d}{1-d}$. Par la suite, on fait les calculs comme vu précédemment. Si vous préférez utilisez les formules en lien avec le taux d'escompte, référez vous à la
    fin du document.
\end{mathfinch1}

\end{multicols*}

\end{document}
