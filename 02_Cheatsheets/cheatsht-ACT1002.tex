\documentclass[10pt, french]{article}
\usepackage[landscape, hmargin=1cm, vmargin=1.7cm]{geometry}

%% -----------------------------
%% Préambule
%% -----------------------------
% !TEX encoding = UTF-8 Unicode
% LaTeX Preamble for all cheatsheets
% Author : Gabriel Crépeault-Cauchon

% HOW-TO : copy-paste this file in the same directory as your .tex file, and add in your preamble the next command right after you have specified your documentclass : 
% \input{preamble-cheatsht.tex}
% ---------------------------------------------
% ---------------------------------------------

% Extra note : this preamble creates document that are meant to be used inside the multicols environment. See the documentation on internet for further information.

%% -----------------------------
%% Encoding packages
%% -----------------------------
\usepackage[utf8]{inputenc}
\usepackage[T1]{fontenc}
\usepackage{babel}
\usepackage{lmodern}
\usepackage[colorinlistoftodos]{todonotes}
%% -----------------------------
%% Variable definition
%% -----------------------------
\def\auteur{\href{https://github.com/ressources-act/Guide_de_survie_en_actuariat/blob/master/02_Cheatsheets/contributeurs/contributeurs-cheatshts.pdf}{\faGithub \ Liste des contributeurs}}
\def\BackgroundColor{white}
\usepackage{xargs} % for more logical new function creation

%% -----------------------------
%% Margin and layout
%% -----------------------------
% Determine the margin for cheatsheet
\usepackage[landscape, hmargin=1cm, vmargin=1.7cm]{geometry}
\usepackage{multicol}

% Remove automatic indentation after section/subsection title.
\setlength{\parindent}{0cm}

% Save space in cheatsheet by removing space between align environment and normal text.
\usepackage{etoolbox}
\newcommand{\zerodisplayskips}{%
  \setlength{\abovedisplayskip}{0pt}%
  \setlength{\belowdisplayskip}{0pt}%
  \setlength{\abovedisplayshortskip}{0pt}%
  \setlength{\belowdisplayshortskip}{0pt}}
\appto{\normalsize}{\zerodisplayskips}
\appto{\small}{\zerodisplayskips}
\appto{\footnotesize}{\zerodisplayskips}

%% -----------------------------
%% URL and links
%% -----------------------------
\usepackage{hyperref}
\hypersetup{colorlinks = true, urlcolor = gray!70!white, linkcolor = black}

%% -----------------------------
%% Document policy (uncomment only one)
%% -----------------------------
%	\usepackage{concrete}
	\usepackage{mathpazo}
%	\usepackage{frcursive} %% permet d'écrire en lettres attachées
%	\usepackage{aeguill}
%	\usepackage{mathptmx}
%	\usepackage{fourier} 

%% -----------------------------
%% Math configuration
%% -----------------------------
\usepackage[fleqn]{amsmath}
\usepackage{amsthm,amssymb,latexsym,amsfonts}
\usepackage{gensymb}
\usepackage{empheq}
\usepackage{numprint}
\usepackage{dsfont} % Pour avoir le symbole du domaine Z
%\usepackage{bigints} % pour des gros intégrales
% Mathematics shortcuts
\usepackage{scalerel,stackengine,amsmath}
\newcommand\equalhat{\mathrel{\stackon[1.5pt]{=}{\stretchto{%
    \scalerel*[\widthof{=}]{\wedge}{\rule{1ex}{3ex}}}{0.5ex}}}}
\newcommand{\reels}{\mathbb{R}}
\newcommand{\entiers}{\mathbb{Z}}
\newcommand{\naturels}{\mathbb{N}}
\newcommand{\eval}{\biggr \rvert}
\usepackage{cancel}
\newcommand{\derivee}[1]{\frac{\partial}{\partial #1}}
\newcommand{\prob}[1]{\Pr \left( #1 \right)}
\newcommand{\esp}[1]{\mathrm{E} \left[ #1 \right]} % espérance
\newcommand{\variance}[1]{\mathrm{Var} \left( #1   \right)}
\newcommand{\covar}[1]{\mathrm{Cov} \left( #1   \right)}
\newcommand{\laplace}{\mathcal{L}}
\newcommand{\deriv}[3][]{\frac{\partial^{#1}#3}{\partial #2^{#1}}}
\newcommand{\e}[1]{\mathrm{e}^{#1}}
\newcommand{\te}[1]{\text{exp}\left\{#1\right\}}
\DeclareMathSymbol{\shortminus}{\mathbin}{AMSa}{"39}
%%	Example usage:	\sumz{n}{i = 1} <=> \overset{n}{\underset{i = 1}{\sum}}
\newcommand{\sumz}[2]{\overset{#1}{\underset{#2}{\sum}}}
%%	Example usage:	\limz{h}{0} <=> \underset{h \rightarrow 0}{\lim}
\newcommand{\limz}[2]{\underset{#1 \rightarrow #2}{\lim}}
%%	Example usage:	\LVx{h}	<=>	\actsymb[h]{L}{}[]
%%					\LVx[n]{h}	<=>	\actsymb[h]{L}{}[n]
\newcommand{\LVx}[2][]{\actsymb[#2]{L}{}[#1]}
\DeclareMathOperator*{\argmax}{arg\,max}
\DeclareMathOperator*{\argmin}{arg\,min}
%%%	\icbox{<frame color>}{<background color>}{<text>}
\newcommandx{\icbox}[3][1 = bleudefrance, 2 = beaublue]{\fcolorbox{#1}{#2}{#3}}
%%	other good color combo is azure(colorwheel) arsenic
\usepackage{longfbox}
%	voir cette page, paquetage avec CSS https://ctan.math.illinois.edu/macros/latex/contrib/longfbox/longfbox.html
\newfboxstyle{rappel}{
	background-color = tealblue!20!white, 
	border-style = outset,
	breakable = true,
%	
	border-color = tealblue,
	border-radius = 1ex, 
%
	padding-bottom = 0.2ex,
	padding-top = 0.2ex,
	padding-left = 0.4ex,
	padding-right = 0.4ex,
%	
	border-top-width = 0.3ex,
	border-bottom-width = 0.3ex,
%
	border-left-width = 1ex, 
	border-bottom-left-radius = 0.2ex,
%	
	border-right-width = 1ex, 
	border-top-right-radius = 0.2ex,
%	
}
\newfboxstyle{formula}{ 
	background-color = beaublue, 
	border-color = bleudefrance
}
\newfboxstyle{imphl}{ 
	padding = 0pt,
	margin = 0pt,
	baseline-skip = false,
	background-color = palechestnut!60!white, 
	border-color = white
}
\newfboxstyle{conditions}{ 
	background-color = palechestnut, 
	border-color = red
}
\newcommandx{\rcbox}[3][1 = bleudefrance, 2 = beaublue]{\lfbox[border-radius = 0.5ex, background-color = #2, border-color = #1]{#3}}

% To indicate equation number on a specific line in align environment
\newcommand\numberthis{\addtocounter{equation}{1}\tag{\theequation}}

%
% Actuarial notation packages
%
\usepackage{actuarialsymbol}
\usepackage{actuarialangle}

%
% Matrix notation for math symbols (\bm{•})
%
\usepackage{bm}
% Matrix notation variable (bold style)
\newcommand{\matr}[1]{\mathbf{#1}}



%% -----------------------------
%% tcolorbox configuration
%% -----------------------------
\usepackage[most]{tcolorbox}
\tcbuselibrary{xparse}
\tcbuselibrary{breakable}

%%
%% Coloured box "definition" for definitions
%%
\DeclareTColorBox{definition}{ o }				% #1 parameter
{
	colframe=black,colback=white, % color of the box
	breakable, 
	pad at break* = 0mm, 						% to split the box
	title = {#1},
	after title = {\large \hfill \faBook},
}
%%
%% Coloured box "definition2" for definitions
%%
\DeclareTColorBox{definitionNOHFILL}{ o }				% #1 parameter
{
	colframe=blue!60!green,colback=blue!5!white, % color of the box
	pad at break* = 0mm, 						% to split the box
	title = {#1},
	before title = {\faBook \quad },
	breakable
}
%%
%% Coloured box "definition2" for definitions
%%
\DeclareTColorBox{definitionNOHFILLsub}{ o }				% #1 parameter
{
	colframe=blue!40!green,colback=blue!5!white, % color of the box
	pad at break* = 0mm, 						% to split the box
	title = {#1},
	before title = {\faNavicon \quad }, %faBars  faGetPocket
	breakable
}
%%
%% Coloured box "definition3" for propriétés
%%
\DeclareTColorBox{definitionNOHFILLprop}{ o }				% #1 parameter
{
	colframe=amber(sae/ece),colback=amber(sae/ece)!5!white, % color of the box
	pad at break* = 0mm, 						% to split the box
	title = {#1},
	before title = {\faGetPocket \quad }, %faBars  faGetPocket
	breakable
}
%%
%% Coloured box "definition3" for propriétés
%%
\DeclareTColorBox{definitionNOHFILLpropos}{ o }				% #1 parameter
{
	colframe=carmine,colback=carmine!5!white, % color of the box
	pad at break* = 0mm, 						% to split the box
	title = {#1},
	before title = {\faColumns \quad }, %\faEllipsisH  faColumns
	breakable
}


%%
%% Coloured box "algo" for algorithms
%%
\newtcolorbox{algo}[ 1 ]
{
	colback = blue!5!white,
	colframe = blue!75!black,
	title=#1,
	fonttitle = \bfseries,
	breakable
}
%%
%% Coloured box "conceptgen" for points adding to a concept's deifintion
%%
\newtcolorbox{conceptgen}[ 1 ]
{
	breakable,
	colback = beaublue,
	colframe = airforceblue,
	title=#1,
	fonttitle = \bfseries
}
%%
%% Coloured box "rappel" pour rappel de formules
%%
\DeclareTColorBox{conceptgen_enhanced}{ o }
{
	enhanced,
	title = #1,
	colback=beaublue, % color of the box
%	colframe=blue(pigment),
%	colframe=arsenic,	
	colbacktitle=airforceblue,
	fonttitle = \bfseries,
	breakable,
	boxed title style={size=small,colframe=arsenic} ,
	attach boxed title to top center = {yshift=-3mm,yshifttext=-1mm},
}
%%
%% Coloured box "probch1" pour formules relatives au 1er chapitre de prob
%%
\newtcolorbox{probch1}[ 1 ]
{
	colback = ao(english)!40!white,
	colframe = forestgreen(traditional),
	fonttitle = \bfseries,	
	breakable,
	title=#1
}
%%
%% Coloured box "probch2" pour formules relatives au 2e chapitre de prob
%%
\newtcolorbox{probch2}[ 1 ]
{
	colback = orange!50!white,
	colframe = burntorange,
	fonttitle = \bfseries,	
	breakable,
	title=#1
}
%%
%% Coloured box "axioms" pour formules relatives à la dernière partie du chapitre 2 de prob
%%
\newtcolorbox{axioms}[ 1 ]
{
	colback = blue!10!white,
	colframe = blue!70!white,
	fonttitle = \bfseries,	
	breakable,
	title=#1
}
%%
%% Coloured box "probch3" pour formules relatives au 3ème chapitre de prob
%%
\newtcolorbox{probch3}[ 1 ]
{
	colback = ruddypink,
	colframe = burgundy,
	fonttitle = \bfseries,	
	breakable,
	title=#1
}
%%
%% Coloured box "formula" for formulas
%%
\newtcolorbox{formula}[ 1 ]
{
	colback = green!5!white,
	colframe = green!70!black,
	breakable,
	fonttitle = \bfseries,
	title=#1
}
%%
%% Coloured box "formula" for formulas
%%
\DeclareTColorBox{algo2}{ o }
{
	enhanced,
	title = #1,
	colback=blue!5!white,	
	colbacktitle=blue!75!black,
	fonttitle = \bfseries,
	breakable,
	boxed title style={size=small,colframe=arsenic} ,
	attach boxed title to top center = {yshift=-3mm,yshifttext=-1mm},
}
%%
%% Coloured box "examplebox" for formulas
%%
\newtcolorbox{examplebox}[ 1 ]
{
	colback = beaublue,
	colframe = amethyst,
	breakable,
	fonttitle = \bfseries,title=#1
}
%%
%% Coloured box "rappel" pour rappel de formules
%%
\newtcolorbox{rappel}[ 1 ]
{
	colback = ashgrey,
	colframe = arsenic,
	breakable,
	fonttitle = \bfseries,title=#1
}
%%
%% Coloured box "rappel" pour rappel de formules
%%
\DeclareTColorBox{rappel_enhanced}{ o }
{
	enhanced,
	title = #1,
	colback=ashgrey, % color of the box
%	colframe=blue(pigment),
%	colframe=arsenic,	
	colbacktitle=arsenic,
	fonttitle = \bfseries,
	breakable,
	boxed title style={size=small,colframe=arsenic} ,
	attach boxed title to top center = {yshift=-3mm,yshifttext=-1mm},
}
%%
%% Coloured box "notation" for notation and terminology
%%
\DeclareTColorBox{distributions}{ o }			% #1 parameter
{
	enhanced,
	title = #1,
	colback=gray(x11gray), % color of the box
%	colframe=blue(pigment),
	colframe=arsenic,	
	colbacktitle=aurometalsaurus,
	fonttitle = \bfseries,
	boxed title style={size=small,colframe=arsenic} ,
	attach boxed title to top center = {yshift=-3mm,yshifttext=-1mm},
	breakable
%	left=0pt,
%  	right=0pt,
%    box align=center,
%    ams align*
%  	top=-10pt
}
\newtcolorbox{contrib}[ 1 ]
{
	colback = babyblueeyes,
	colframe = airforceblue,
	fonttitle = \bfseries,
	title = {#1},
	valign = center
}

%% -----------------------------
%% Graphics and pictures
%% -----------------------------
\usepackage{graphicx}
\usepackage{pict2e}
\usepackage{tikz}

%% -----------------------------
%% insert pdf pages into document
%% -----------------------------
\usepackage{pdfpages}

%% -----------------------------
%% Color configuration
%% -----------------------------
\usepackage{color, soulutf8, colortbl}


%
%	Colour definitions
%
\definecolor{armygreen}{rgb}{0.29, 0.33, 0.13}	%	army
\definecolor{asparagus}{rgb}{0.53, 0.66, 0.42}	% pastel green militariesque
\definecolor{britishracinggreen}{rgb}{0.0, 0.26, 0.15}
\definecolor{calpolypomonagreen}{rgb}{0.12, 0.3, 0.17}
\definecolor{darkgreen}{rgb}{0.0, 0.2, 0.13}
\definecolor{lightgreen}{rgb}{0.2, 0.95, 0.2}

\definecolor{antiquebrass}{rgb}{0.8, 0.58, 0.46}	% brown-ish light cardboard color

\definecolor{blue(munsell)}{rgb}{0.0, 0.5, 0.69}
\definecolor{blue(matcha)}{rgb}{0.596, 0.819, 1.00}
\definecolor{blue(munsell)-light}{rgb}{0.5, 0.8, 0.9}
\definecolor{bleudefrance}{rgb}{0.19, 0.55, 0.91}
\definecolor{blizzardblue}{rgb}{0.67, 0.9, 0.93}	%	mr.freeze light baby blue 
\definecolor{bondiblue}{rgb}{0.0, 0.58, 0.71}	%	darker cyan type inidgo blue
\definecolor{blue(pigment)}{rgb}{0.2, 0.2, 0.6}
\definecolor{bluebell}{rgb}{0.64, 0.64, 0.82}
\definecolor{airforceblue}{rgb}{0.36, 0.54, 0.66}
\definecolor{beaublue}{rgb}{0.74, 0.83, 0.9}    % almost white
\definecolor{blue_rectangle}{RGB}{83, 84, 244}		% ACT-2004
\definecolor{cobalt}{rgb}{0.0, 0.28, 0.67}	% nice light blue-ish
\definecolor{ballblue}{rgb}{0.13, 0.67, 0.8}	%	almost green ish blue ish
\definecolor{babyblueeyes}{rgb}{0.63, 0.79, 0.95}

\definecolor{indigo(web)}{rgb}{0.29, 0.0, 0.51}	% purple-ish
\definecolor{antiquefuchsia}{rgb}{0.57, 0.36, 0.51}	%	pastel matte (darkerish) purple ish
\definecolor{darkpastelpurple}{rgb}{0.59, 0.44, 0.84}	%	pretty purple
\definecolor{gray(x11gray)}{rgb}{0.75, 0.75, 0.75}
\definecolor{aurometalsaurus}{rgb}{0.43, 0.5, 0.5}
\definecolor{bulgarianrose}{rgb}{0.28, 0.02, 0.03}	%	dark maroon type 
\definecolor{pastelred}{rgb}{1.0, 0.41, 0.38}		%	light red pinktinybit ish
\definecolor{lightmauve}{rgb}{0.86, 0.82, 1.0}
\definecolor{eggshell}{rgb}{0.94, 0.92, 0.84}
\definecolor{azure(colorwheel)}{rgb}{0.0, 0.5, 1.0}
\definecolor{darkgreen}{rgb}{0.0, 0.2, 0.13}			
\definecolor{ao(english)}{rgb}{0.0, 0.5, 0.0}		% prertty apple dark pastel (light) green
\definecolor{green_rectangle}{RGB}{131, 176, 84}		% ACT-2004
\definecolor{red_rectangle}{RGB}{241,112,113}		% ACT-2004
\definecolor{amethyst}{rgb}{0.6, 0.4, 0.8}
\definecolor{amethyst-light}{rgb}{0.6, 0.4, 0.8}
\definecolor{ruddypink}{rgb}{0.88, 0.56, 0.59}

\definecolor{amber(sae/ece)}{rgb}{1.0, 0.49, 0.0} 	%	pretty orange ish
\definecolor{burntsienna}{rgb}{0.91, 0.45, 0.32}		%%	lighter pastel orange
\definecolor{burntorange}{rgb}{0.8, 0.33, 0.0}		%%	similar but deeper orange
\definecolor{orange-red}{rgb}{1.0, 0.27, 0.0}

\definecolor{tealblue}{rgb}{0.21, 0.46, 0.53}

\definecolor{battleshipgrey}{rgb}{0.52, 0.52, 0.51}  % lilght ish gray
\definecolor{ashgrey}{rgb}{0.7, 0.75, 0.71}			% dark grey-black-ish
\definecolor{arsenic}{rgb}{0.23, 0.27, 0.29}			% light green-beige-ish gray
\definecolor{gray(x11gray)}{rgb}{0.75, 0.75, 0.75}

\definecolor{carmine}{rgb}{0.59, 0.0, 0.09} 			% deep red
\definecolor{amaranth}{rgb}{0.9, 0.17, 0.31}
\definecolor{brickred}{rgb}{0.8, 0.25, 0.33}
\definecolor{chestnut}{rgb}{0.8, 0.36, 0.36}		% pink red ish light
\definecolor{palechestnut}{rgb}{0.87, 0.68, 0.69}
\definecolor{pastelred}{rgb}{1.0, 0.41, 0.38}
\definecolor{forestgreen(traditional)}{rgb}{0.0, 0.27, 0.13}
%
% Useful shortcuts for coloured text
%
\newcommand{\orange}{\textcolor{orange}}
\newcommand{\red}{\textcolor{red}}
\newcommand{\cyan}{\textcolor{cyan}}
\newcommand{\blue}{\textcolor{blue}}
\newcommand{\green}{\textcolor{green}}
\newcommand{\purple}{\textcolor{magenta}}
\newcommand{\yellow}{\textcolor{yellow}}

%% -----------------------------
%% Enumerate environment configuration
%% -----------------------------
%
% Custum enumerate & itemize Package
%
\usepackage{enumitem}
%
% French Setup for itemize function
%
\frenchbsetup{StandardItemLabels=true}
%
% Change default label for itemize
%
\renewcommand{\labelitemi}{\faAngleRight}


%% -----------------------------
%% Tabular column type configuration
%% -----------------------------
\newcolumntype{C}{>{$}c<{$}} % math-mode version of "l" column type
\newcolumntype{L}{>{$}l<{$}} % math-mode version of "l" column type
\newcolumntype{R}{>{$}r<{$}} % math-mode version of "l" column type
\newcolumntype{f}{>{\columncolor{green!20!white}}p{1cm}}
\newcolumntype{g}{>{\columncolor{green!40!white}}m{1.2cm}}
\newcolumntype{a}{>{\columncolor{red!20!white}$}p{2cm}<{$}}	% ACT-2005
% configuration to force a line break within a single cell
\usepackage{makecell}


%% -----------------------------
%% Fontawesome for special symbols
%% -----------------------------
\usepackage{fontawesome}

%% -----------------------------
%% Section Font customization
%% -----------------------------
\usepackage{sectsty}
\sectionfont{\color{\SectionColor}}
\subsectionfont{\color{\SubSectionColor}}
\subsubsectionfont{\color{\SubSubSectionColor}}

%% -----------------------------
%% Footer/Header Customization
%% -----------------------------
\usepackage{lastpage}
\usepackage{fancyhdr}
\pagestyle{fancy}

%
% Header
%
\fancyhead{} 	% Reset
\fancyhead[L]{Aide-mémoire pour~ \cours ~(\textbf{\sigle})}
\fancyhead[R]{\auteur}

%
% Footer
%
\fancyfoot{}		% Reset
\fancyfoot[R]{\thepage ~de~ \pageref{LastPage}}
\fancyfoot[L]{\href{https://github.com/ressources-act/Guide_de_survie_en_actuariat}{\faGithub \ ressources-act/Guide de survie en actuariat}}
%
% Page background color
%
\pagecolor{\BackgroundColor}




%% END OF PREAMBLE
% ---------------------------------------------
% ---------------------------------------------
%% -----------------------------
%% Variable definition
%% -----------------------------
%% -----------------------------
%% Footer and header customization
%% -----------------------------
\def\cours{Analyse probabiliste des risques actuariels}
\def\sigle{ACT-1002}
\fancyfoot[R]{\thepage ~de~ \pageref{LastPage}}
\setlist{leftmargin=*}


%% -----------------------------
%% Colour setup for sections
%% -----------------------------
\def\SectionColor{green!50!black}
\def\SubSectionColor{green!20!black}
%% -----------------------------
%% Definition of LaTex math commands
%% -----------------------------
\newcommand{\norm}[1]{\left\lVert#1\right\rVert}
%
% Redefine authors
%
\definecolor{burgundy}{rgb}{0.5, 0.0, 0.13}

\begin{document}
\begin{center}
	\textsc{\Large Contributeurs}\\[0.5cm] 
\end{center}
\input{contributeurs/contrib-ACT1002}

\newpage

\raggedcolumns
\begin{multicols*}{3}

\section{Chapitre 1: Analyse combinatoire}
\begin{probch1}{L'analyse combinatoire}
\begin{description}
  \item[Principe de base de comptage :] Pour une expérience 1 avec m résultats possibles et une expérience 2 avec n résultats possibles, il y a m x n possibilités.
  \item[Permutations :] Nombre d'arrangements en tenant compte de l'ordre. La façon de dénombrer les arrangements dépend du type de question.
	  \item[$Exemples$ :] Combien de façons peut-on arranger les chiffres 1, 2, 3 et 4 dans un nombre à 4 chiffres? La réponse est {4!}. La réponse serait la même avec les chiffres 1, 2, 2 et 3, car les deux chiffres 2 ne sont pas considérés comme identiques. 
	  \item[] Toutefois, si on demande combien de façons peut-on distribuer 12 cadeaux différents à 4 personnes, la réponse sera $ 4^{12} $ étant donné que chaque cadeau peut aller à quatre personnes.
	\item[Combinaisons :] Nombre d'arrangements en \textbf{ne} tenant \textbf{pas} compte de l'ordre.
	 \item[$Exemples$ :] En reprenant l'exemple des permutations, mais avec les chiffres 1, 1, 1, 2 et 2, il y aurait 5!/(3!2!) combinaisons puisque les trois 1 et les deux 2 sont considérés identiques.
	 \item[Coefficient binomial :] De l'exemple précédent, on peut observer l'existence du coefficient binomial, qui est défini selon la formule :
	 \begin{align*}
	 \binom{n}{k}
	 &= \frac{n!}{k! \,(n-k)!}
	 \end{align*}
  \item[Coefficient multinomial :] La généralisation du coefficient binomial va comme suit :
  \begin{align*}
  \binom{n}{k_1, k_2, ..., k_m}
  &= \frac{n!}{(k_1)! \, (k_2)! \, ..., (k_m)!}
  \end{align*}
  \item[] Où {$n = \sum_{i = 1}^{m} k_i$}
<<<<<<< Updated upstream
  \item[Théorème multinomial :] Le coefficient multinomial aide à trouver les coefficients devant les variables lors du développement de multinômes. 
  \begin{itemize}
	  \item Afin de mieux comprendre le théorème multinomial, voici un exemple :
	  	\setlength{\mathindent}{-1cm}
		  \begin{align*}
			(-3x+5y^2)^4 
			&=	\sum_{}^{} \binom{4}{n_1 , n_2} (-3x)^{n_1} (5y^{2})^{n_2}
		  \end{align*}
		\setlength{\mathindent}{1cm}
	 \item Si on cherche le coefficient devant les variables $x^{3}y^{2}$, on remplace $n_1$ par 3 et $n_2$ par 1 et on obtient -540.
\end{itemize} 
\end{description}  
\begin{description}
  \item[Solutions entières non-négatives :] Le nombre de façons dont on peut distribuer un nombre d'objets indissociables dans des «contenants». 
	\begin{itemize}
		\item Il peut y avoir aucun objet dans un «contenant». 
		\item La solution à ce type de problème est donnée par $\binom{n+r-1}{r-1}$ où n est le nombre d'objets et r le nombre de «contenants». 
		\item S'il est mentionné dans le problème qu'il n'est pas nécessaire d'utiliser tous les objets pour les mettre dans les contenants, on peut rajouter un contenant pour les objets «non-utilisés».
	\end{itemize}
  \item[Solutions entières positives:] Le nombre de façons dont on peut distribuer un nombre d'objets indissociables dans des «contenants» et ce, de façon à ce que chaque «contenant» ait au moins un objet. 
  \begin{itemize}
	  \item La solution à ce type de problème est donnée par $\binom{n-1}{r-1}$.
  \end{itemize} 
=======
  \item[Théorème multinomial :] Le coefficient multinomial aide à trouver les coefficients devant les variables lors du développement de multinômes. Afin de mieux comprendre le théorème multinomial, voici un exemple :
  \item[]{$(-3x+5y^2)^4 = \sum_{}^{} \binom{4}{n_1 , n_2} (-3x)^{n_1} (5y^{2})^{n_2}$}
  \item[] Si on cherche le coefficient devant les variables $x^{3}y^{2}$, on remplace $n_1$ par 3 et $n_2$ par 1 et on obtient -540.
  \item[Solutions entières non-négatives :] Le nombre de façons dont on peut distribuer un nombre d'objets indissociables dans des «contenants». Il peut y avoir aucun objet dans un «contenant». La solution à ce type de problème est donnée par $\binom{n+r-1}{r-1}$ où n est le nombre d'objets et r le nombre de «contenants». Si l'est mentionné dans le problème qu'il n'est pas nécessaire d'utiliser tous les objets pour les mettre dans les contenants, on peut rajouter un contenant pour les objets «non-utilisés».
  \item[Solutions entières positives:] Le nombre de façons dont on peut distribuer un nombre d'objets indissociables dans des «contenants» et ce, de façon à ce que chaque «contenant» ait au moins un objet. La solution à ce type de problème est donnée par $\binom{n-1}{r-1}$.
>>>>>>> Stashed changes
\end{description}
\end{probch1}

\pagebreak
\section{Chapitre 2: Axiomes de probabilité}
%faut trouver une façon de ramener les colonnes à gaucher comme dans probch1 FC
% effectivement,  pas encore trouvé comment (AJvR)
\begin{rappel}{Domaine et définition}
\begin{description}
	\item[Random Process:]	Famille de variable aléatoires $\{X_{t}: t \in T\}$ qui associe un espace d'états $\Omega$ à un ensemble $S$.
	\begin{enumerate}
		\item[$\Omega$: ] L'espace d'états $\Omega$ est composé des événements possibles de la variable aléatoire $X$.
		\item[] Par exemple, lorsqu'on lance une pièce de monnaie $\Omega = \{\textsf{Face}, \texttt{Pile} \}$.
		\item[$S$: ] L'ensemble $S$ est l'ensemble des probabilités des événements dans $\Omega$.
		\item[] Par exemple, lorsqu'on lance une pièce de monnaie $S = \{\frac{1}{2}, \frac{1}{2} \}$.
	\end{enumerate}
\end{description}
\begin{description}
	\item[iid: ]	Les variables aléatoire $X_{t}$ doivent être indépendantes et identiquement distribuées. Ceci est dénoté par \textit{\textbf{i.i.d.}}.				
	\begin{description}
		\item[indépendant: ] Si $X_{t}$ est une variable aléatoire \textit{iid} alors, pour 2 variables aléatoires $X_{i}$ et $X_{j}$, où $i, j \in T$, le résultat de $X_{i}$ n'a aucun impact sur le résultat de $X_{j}$ pour tout $t \in T$.
		\item[identiquement distribué: ] L'ensemble $S$ est l'ensemble des probabilités des événements dans $\Omega$.
	\end{description}
	\item[Probabilité de $X_{t}$: ]	La probabilité d'un événement $X_{t}$ est dénoté $\Pr(X_{t})$. 
	Ces probabilités forment l'ensemble $S$.
	\begin{description}
		\item[Propriété: ] $\sum_{i = -\infty}^{\infty} \Pr(X_{i}) = 1$
	\end{description}
	\item[Types de variables aléatoire: ]	Il y a 2 types de variables aléatoire, les distributions \textit{discrètes} et \textit{continues}.
		\begin{description}
		\item[Discrète: ]	Si l'ensemble $S$ est dénombrable, c'est-à-dire que $S= \{s\}$, alors la variable aléatoire $X$ est dite \textbf{discrète}.
		\item[Continue: ]	Si l'ensemble $S$ n'est pas dénombrable alors la variable aléatoire $X$ est dite \textbf{continue}.
	\end{description}
\end{description}
\end{rappel}

\begin{probch2}{Concepts et opérations sur les ensembles}
\begin{description} 

%ce serait bien d'avoir des diagrammes de Venn pour des exemples, ce serait plus visuel et surtout moins encombré FC

  \item[L'union ({$\cup$}) :] On peut le définir par un ou. 
  	\begin{itemize}
  	\item	Si l'événement A est d'avoir 3 sur un dé et l'événement B est d'avoir 4 sur ce même dé, les résultats possibles de A{$\cup$}B est 3 et 4.
  	\end{itemize}
  \item[L'intersection ({$\cap$}) :] On peut le définir par un et. 
  	\begin{itemize}
  	\item	Si l'événement A est d'avoir un chiffre pair sur un dé et que l'événement B est d'avoir 5 ou 6 sur ce même lancer de dé, le résultat de $A \cap B$ est 6, car 6 est un nombre pair et fait partie de l'ensemble B. 
  	\end{itemize}
  \item[Complémentaire :] Un événement quelconque est le complémentaire d'un événement A lorsqu'il correspond à tous les résultats de $\Omega$ excluant les résultats de A. 
  	\begin{itemize}
  	\item	Un exemple est l'événement «Avoir un nombre pair sur un dé»; un événement complémentaire serait donc «Avoir un nombre impair sur un dé». 
  	\item	Le complémentaire d'un événement A est désigné par {$\overline{\rm A}$}, $A^{c}$ et $A^{t}$.
  	\end{itemize}
  %C'est quoi le nom des symboles qui suivent, cad la somme d'unions et d'intersections? 
  \item[«Somme d'unions»($\bigcup_{i=1}^{n} A_i$) :] Représentation plus simple et courte de $A_1 cup A_2 \cup \dots \cup A_n$
  \item[«Somme d'intersections»($\bigcap_{i=1}^{n} A_i$) :] Représentation plus simple et courte de $A_1 \cap A_2 \cap \dots \cap A_n$
  \item[Opérations sur les ensembles :] Les événements peuvent agir à un certain point comme des termes mathématiques, c'est-à-dire qu'on peut effectuer des opérations avec ceux-ci.
  \item[Commutativité :] $A_1	\cup	A_2	=	A_2 \cup A_1$
  \item[] $A_1	\cap	A_2	=	A_2 \cap A_1$
  \item[Associativité :] ($(A_1 \cup A_2)	\cup A_3	=	A_1 \cup		(A_2 \cup A_3)$)
  \item[] ($(A_1 \cap A_2)	\cap A_3	=	A_1 \cap		(A_2 \cap A_3)$)
  \item[Distributivité :] $(A_1 \cup A_2)	\cap A_3 = (A_1 \cap A_3) \cup (A_2 \cap A_3)$)
  \item[] $(A_1 \cap A_2)	\cup A_3 = (A_1 \cup A_3) \cap (A_2 \cup A_3)$)
  \item[Loi de DeMorgan :] $(\bigcup_{i = 1}^{n} A_{i})^{c} = (\bigcap_{i=1}^{n} A_i^{c})$
  \item[] $(\bigcap_{i = 1}^{n} A_{i})^{c} = (\bigcup_{i=1}^{n} A_i^{c})$
\end{description}
\end{probch2}

\begin{axioms}{Axiomes de probabilité}
\begin{description}
  \item[Définition :] Des axiomes de probabilités sont en quelque sorte des règles, des contraintes ou des formules relatives aux probabilités.
\end{description}
\end{axioms}{Axiomes de probabilité}

\pagebreak
\section{Chapitre 3: Probabilité conditionnelle}
\begin{probch3}{Probabilité conditionnelle}
\begin{description}
	\item[Conditionnel:]	La probabilité que $A$ arrive \textit{sachant} que $B$ est arrivé est: 
	\begin{align*}
		\Pr(A | B) 
		&= 	\frac{\Pr(A \cap B)}{\Pr(B)}	
	\end{align*}
	où la probabilité que $B$ arrive est non-nulle, $\Pr(B) > 0$.
	\item[Indépendant:]	Les événements $A$ et $B$ sont indépendant si:
	\begin{equation*}
		\Pr(A \cap B) = \Pr(A) \cdot \Pr(B)
	\end{equation*}
\end{description}
Avec la première définition de la probabilité conditionnelle, on peut trouver ces résultats:
\begin{description}
	\item[Relation probabilité conditionnelle: ]	La probabilité que l'événement $E_{2}$ ai lieu sachant que l'événement $E_{1}$ à déjà eu lieu est équivalent à la probabilité que l'événement $E_{1}$ ai lieu sachant que $E_{2}$ à \textit{déjà} eu lieu multiplié par la probabilité que l'événement $E_{2}$ ai lieu peu importe $E_{1}$. \\
	Le tout est encore pondéré par la probabilité que l'événement $E_{1}$ ai lieu peu importe si $E_{2}$ y a.
	\begin{align*}
		\Pr(E_{2} | E_{1})
		&= 	\frac{\Pr(E_{2} \cap E_{1})}{\Pr(E_{1})}		\\
		&=	\frac{\Pr(E_{1} | E_{2}) \Pr(E_{2})}{\Pr(E_{1})}		
	\end{align*}
	\item[Loi des probabilités totales: ]	Les probabilités liées à la variable aléatoire $E$ lorsqu'elles sont conditionnelles à la variable aléatoire discrète $F$ est dénoté comme suit:
	\begin{align*}
		\Pr(E)	&=	\sum_{i = 1}^{n} \Pr(E | F_{i}) \Pr(F_{i})
	\end{align*}
	\item[Formule de Bayes: ]	On combine les deux résultats précédent:
	\begin{align*}
		\Pr(F_{i} | E)
		&=	\frac{\Pr(E | F_{i})}{\sum_{i = 1}^{n} \Pr(E | F_{i}) \Pr(F_{i})}
	\end{align*}
\end{description}
\end{probch3}

\end{multicols*}

\end{document}
